\documentclass[12pt]{report}
\usepackage[english, magyar]{babel}
\usepackage{t1enc}
\frenchspacing

\usepackage[margin=2cm, top=5cm, bottom=2.5cm, bindingoffset=0cm]{geometry}
\usepackage{graphicx}

\usepackage{hyperref}
\hypersetup{hidelinks}

\usepackage{xcolor,listings}
\usepackage{textcomp}
\usepackage{color}
\usepackage{listingsutf8}

\renewcommand{\lstlistingname}{Programkód}

\definecolor{codegreen}{rgb}{0,0.6,0}
\definecolor{codegray}{rgb}{0.5,0.5,0.5}
\definecolor{codepurple}{HTML}{C42043}
\definecolor{backcolour}{HTML}{F2F2F2}
\definecolor{bookColor}{cmyk}{0,0,0,0.90}  
\definecolor{xmltagcolor}{rgb}{0,0,1}
\definecolor{xmlcommentcolor}{rgb}{0,0.6,0}
\definecolor{xmlstringcolor}{rgb}{0.6,0,0}
\color{bookColor}
\lstset{upquote=true}

\lstdefinestyle{mystyle}{
	language=XML,
	basicstyle=\ttfamily\footnotesize,
	morestring=[b]",
	morestring=[s][\color{xmltagcolor}]{<}{>},
	morecomment=[s][\color{xmlcommentcolor}]{<!--}{-->},
	showstringspaces=false,
	breaklines=true,
	breakatwhitespace=true,
	tabsize=2,
	captionpos=b,
	extendedchars=true,
	keepspaces=true,
}

\lstset{style=mystyle}

\usepackage{fancyhdr}
\fancypagestyle{plain}{
\fancyhf{}
\fancyhead[R]{\leftmark}
\fancyhead[L]{\thepage}
\fancyfoot[C]{Adatkezelés XML környezetben}}

\fancyhead[R]{\leftmark}
\fancyhead[L]{\thepage}
\fancyfoot[C]{Adatkezelés XML környezetben}

\begin{document}
	\pagestyle{fancy}
	\title{\Huge JEGYZŐKÖNYV \\ \LARGE Adatkezelés XML környezetben}
	\author{\Large Féléves feladat: Sport Club}
	\date{\vspace{250px}
		\begin{flushleft}
			Készítette: \textbf{Takács Ákos}\\
			Neptunkód: \textbf{VTJ4ES}\\
			Dátum: \textbf{2025. 12. 03.}
		\end{flushleft}
		\vspace{15px}
		\begin{center}
			\textbf{Miskolc, 2025}
		\end{center}}
	\maketitle
	
\tableofcontents
\clearpage

\fancyhf{}
\fancyhead[L]{\thepage}

\chapter{A feladat leírása}

A feladat egy sport club (sportegyesület) adatainak modellezése és XML alapú megvalósítása. A rendszerben megjelennek az egyesületi sportolók, az edzők, az edzéscsoportok, a sportlétesítmények, a gyakorlott sportágak, valamint a tagsági viszonyok. A cél egy olyan XML alapú adatleírás kialakítása, amely később Java DOM API segítségével olvasható, módosítható, lekérdezhető és fájlba kiírható.

A tervezés és a megvalósítás során az angol nyelvű elnevezéseket használtam (entitásnevek, elemek, attribútumok), mivel ez a gyakorlat elterjedt a szoftverfejlesztésben.

Összesen 6 egyedet hoztam létre, melyek a következők:

\begin{itemize}
	\item Athlete,
	\item Coach,
	\item TrainingGroup,
	\item Facility,
	\item Sport,
	\item Membership.
\end{itemize}

A gyökérből kiinduló központi logikai entitás a \textbf{Sport\_Club\_VTJ4ES} dokumentum, amely tartalmazza a felsorolt entitások példányait.

\textbf{Athlete}: a sportolókat leíró egyed. Elsődleges kulcsa az \texttt{athleteID} attribútum. A sportolóhoz név (vezetéknév, keresztnév), születési dátum, elérhetőségi adatok (e-mail, telefonszám), valamint többértékű \texttt{skills} (pl. speed, endurance, technique) tulajdonság tartozik.

\textbf{Coach}: az edzőket tartalmazó egyed. Elsődleges kulcsa a \texttt{coachID}. Az edző neve, tapasztalati évei, valamint többértékű specializációs lista (\texttt{specializations}) jelenik meg, például \texttt{rehab}, \texttt{mobility}, \texttt{endurance training}.

\textbf{TrainingGroup}: az edzéscsoportokat leíró egyed. Elsődleges kulcsa a \texttt{groupID}. A csoportnak van neve, szintje (pl. beginner, advanced) és egy többértékű \texttt{schedule} tulajdonsága (hét napja, kezdési és befejezési idő). A csoport egy adott edzőhöz, létesítményhez és sportághoz kapcsolódik, ezt az \texttt{coachRef}, \texttt{facilityRef} és \texttt{sportRef} idegen kulcs attribútumok biztosítják.

\textbf{Facility}: a sportlétesítményeket reprezentálja. Elsődleges kulcsa a \texttt{facilityID}. Tartalmaz nevet, címet (irányítószám, város, utca, házszám) és \texttt{capacity} attribútumot, amely a maximális befogadóképességet mutatja.

\textbf{Sport}: a gyakorlott sportágakat leíró egyed. Elsődleges kulcsa a \texttt{sportID}. Tartalmazza a sportág nevét, kategóriáját (pl. combat, team, individual) és egy \texttt{is\_indoor} logikai értéket, amely jelzi, hogy elsősorban beltéri sportágról van-e szó.

\textbf{Membership}: a sportolók és edzéscsoportok közötti \texttt{N:M} kapcsolatot leíró egyed. Elsődleges kulcsa a \texttt{membershipID}. Idegen kulcs attribútumként tárolja az \texttt{athleteRef} és \texttt{groupRef} értékeket, valamint további tulajdonságként a tagság \texttt{start\_date}, \texttt{status} és opcionálisan \texttt{fee} adatait.

Kapcsolatok összefoglalva:

\begin{itemize}
	\item Egy \textbf{Coach} több \textbf{TrainingGroup}-ot is vezethet, de egy csoporthoz pontosan egy edző tartozik: \texttt{Coach~1:N~TrainingGroup}.
	\item Egy \textbf{Facility} több \textbf{TrainingGroup} edzéseinek helyszíne lehet: \texttt{Facility~1:N~TrainingGroup}.
	\item Egy \textbf{Sport} több \textbf{TrainingGroup}-hoz is kapcsolódhat: \texttt{Sport~1:N~TrainingGroup}.
	\item Egy \textbf{Athlete} több \textbf{TrainingGroup}-ban is tag lehet, és egy csoportnak is több sportoló tagja lehet, így \textbf{Membership} közvetíti az \texttt{Athlete} és \texttt{TrainingGroup} közötti \texttt{N:M} kapcsolatot.
\end{itemize}

\chapter{I. feladat - XML/XSD létrehozás}

\section[ER modell]{A feladat ER modellje}

\begin{figure}[h]
	\centering
	\includegraphics[width=0.999\linewidth]{ERVTJ4ES.png}
	\caption{A sport club ER modellje}
\end{figure}

\section[XDM modell]{A feladat XDM modellje}

A XDM modell kialakításakor az ER modellben definiált entitásokat és kapcsolatokat kellett átvezetni XML-re. Figyelembe vettem az \texttt{1:1}, \texttt{1:N} és \texttt{N:M} kapcsolatokat, valamint az entitások elsődleges kulcsait.

Az \texttt{1:N} kapcsolatoknál a szaggatott nyíl mindig a ,,több'' oldalon lévő kulcshoz mutat. Például a \textbf{Coach} és \textbf{TrainingGroup} közötti kapcsolat esetén a \texttt{groupID} felől mutat a \texttt{coachID}-re hivatkozó \texttt{coachRef} attribútumhoz. Az \texttt{N:M} kapcsolat (Membership) esetében külön XDM modell (kapcsoló entitás) jelenik meg, amely tartalmazza a saját elsődleges kulcsát és a hivatkozásokat mindkét fő entitásra.

Többértékű tulajdonságok (pl. \texttt{skills}, \texttt{specializations}, \texttt{schedule}) XML-ben belső, ismétlődő elemekkel jelennek meg. A XDM modell gyökéreleme: \textbf{Sport\_Club\_VTJ4ES}.

\begin{figure}[h]
	\centering
	\includegraphics[width=1.01\linewidth]{XDMVTJ4ES.png}
	\caption{A sport club XDM modellje}
\end{figure}

\section[Az XML dokumentum]{Az XDM modell alapján XML dokumentum készítése}

Az \texttt{Sport\_Club\_VTJ4ES.xml} dokumentum \texttt{XML 1.0} szabvány szerint készült, \texttt{UTF--8} kódolással. A dokumentum tetején hivatkozom az XSD sémára, amely a szerkezetet és a típusmegkötéseket írja le.

\begin{lstlisting}[caption={A sport club XML dokumentum}]
<?xml version="1.0" encoding="UTF-8"?>

<Sport_Club_VTJ4ES xmlns:xsi="http://www.w3.org/2001/XMLSchema-instance"
                   xsi:noNamespaceSchemaLocation="SportClub_VTJ4ES.xsd">

  <!-- ATHLETE peldanyok -->
  <Athlete athleteID="1">
    <name>
      <first_name>Daniel</first_name>
      <last_name>Kiss</last_name>
    </name>
    <birth_date>2001-04-10</birth_date>
    <contacts>
      <e_mail>daniel.kiss@example.com</e_mail>
      <phone_number>+36-30-111-1111</phone_number>
    </contacts>
    <skills>speed</skills>
    <skills>endurance</skills>
    <skills>technique</skills>
  </Athlete>

  <Athlete athleteID="2">
    <name>
      <first_name>Bence</first_name>
      <last_name>Nagy</last_name>
    </name>
    <birth_date>1999-09-21</birth_date>
    <contacts>
      <e_mail>bence.nagy@example.com</e_mail>
      <phone_number>+36-30-222-2222</phone_number>
    </contacts>
    <skills>strength</skills>
    <skills>agility</skills>
  </Athlete>

  <Athlete athleteID="3">
    <name>
      <first_name>Eszter</first_name>
      <last_name>Kazai</last_name>
    </name>
    <birth_date>2003-02-15</birth_date>
    <contacts>
      <e_mail>eszter.kazai@example.com</e_mail>
      <phone_number>+36-30-333-3333</phone_number>
    </contacts>
    <skills>technique</skills>
  </Athlete>

  <!-- COACH peldanyok -->
  <Coach coachID="1">
    <name>
      <first_name>Laszlo</first_name>
      <last_name>Kovacs</last_name>
    </name>
    <experience_years>15</experience_years>
    <specializations>
      <specialization>boxing</specialization>
      <specialization>kickboxing</specialization>
    </specializations>
  </Coach>

  <Coach coachID="2">
    <name>
      <first_name>Janos</first_name>
      <last_name>Toth</last_name>
    </name>
    <experience_years>8</experience_years>
    <specializations>
      <specialization>muay thai</specialization>
      <specialization>conditioning</specialization>
    </specializations>
  </Coach>

  <Coach coachID="3">
    <name>
      <first_name>Anna</first_name>
      <last_name>Balogh</last_name>
    </name>
    <experience_years>6</experience_years>
    <specializations>
      <specialization>judo</specialization>
    </specializations>
  </Coach>

  <!-- FACILITY peldanyok -->
  <Facility facilityID="1">
    <name>Central Gym Hall</name>
    <address>
      <post_code>3527</post_code>
      <city>Miskolc</city>
      <street>Sport utca</street>
      <number>10</number>
    </address>
    <capacity>120</capacity>
  </Facility>

  <Facility facilityID="2">
    <name>Boxing Room</name>
    <address>
      <post_code>3525</post_code>
      <city>Miskolc</city>
      <street>Kossuth utca</street>
      <number>5</number>
    </address>
    <capacity>40</capacity>
  </Facility>

  <Facility facilityID="3">
    <name>Conditioning Hall</name>
    <address>
      <post_code>3529</post_code>
      <city>Miskolc</city>
      <street>Edzo ter</street>
      <number>2</number>
    </address>
    <capacity>60</capacity>
  </Facility>

  <!-- SPORT peldanyok -->
  <Sport sportID="1">
    <name>Boxing</name>
    <category>combat</category>
    <is_indoor>true</is_indoor>
  </Sport>

  <Sport sportID="2">
    <name>Kickboxing</name>
    <category>combat</category>
    <is_indoor>true</is_indoor>
  </Sport>

  <Sport sportID="3">
    <name>Judo</name>
    <category>combat</category>
    <is_indoor>true</is_indoor>
  </Sport>

  <!-- TRAININGGROUP peldanyok -->
  <TrainingGroup groupID="1" coachRef="1" facilityRef="2" sportRef="1">
    <name>Beginner Boxing</name>
    <level>beginner</level>
    <schedule>
      <session day="Monday" from="18:00" to="19:30"/>
      <session day="Wednesday" from="18:00" to="19:30"/>
    </schedule>
  </TrainingGroup>

  <TrainingGroup groupID="2" coachRef="2" facilityRef="1" sportRef="2">
    <name>Advanced Kickboxing</name>
    <level>advanced</level>
    <schedule>
      <session day="Tuesday" from="19:00" to="20:30"/>
      <session day="Thursday" from="19:00" to="20:30"/>
    </schedule>
  </TrainingGroup>

  <TrainingGroup groupID="3" coachRef="3" facilityRef="3" sportRef="3">
    <name>Judo Juniors</name>
    <level>intermediate</level>
    <schedule>
      <session day="Friday" from="17:00" to="18:30"/>
    </schedule>
  </TrainingGroup>

  <!-- MEMBERSHIP peldanyok (N:M kapcsolat Athlete es TrainingGroup kozott) -->
  <Membership membershipID="1" athleteRef="1" groupRef="1">
    <start_date>2024-01-10</start_date>
    <status>active</status>
    <fee>15000</fee>
  </Membership>

  <Membership membershipID="2" athleteRef="2" groupRef="2">
    <start_date>2023-09-01</start_date>
    <status>active</status>
    <fee>18000</fee>
  </Membership>

  <Membership membershipID="3" athleteRef="3" groupRef="1">
    <start_date>2024-03-15</start_date>
    <status>pending</status>
    <fee>15000</fee>
  </Membership>

</Sport_Club_VTJ4ES>
\end{lstlisting}

\clearpage

\section{Az XML dokumentum alapján XMLSchema készítése}

Az \texttt{SportClub\_VTJ4ES.xsd} séma írja le a sport club XML dokumentum szerkezetét és a megkötéseket. Meghatározza az egyszerű és összetett típusokat, az elsődleges kulcsokat (\texttt{xs:key}) és az idegen kulcs hivatkozásokat (\texttt{xs:keyref}).

\begin{lstlisting}[caption={A sport club XSD dokumentum}]
<?xml version="1.0" encoding="UTF-8"?>
<xs:schema xmlns:xs="http://www.w3.org/2001/XMLSchema">

  <!-- Egyedi egyszeru tipusok -->
  <xs:simpleType name="dateType">
    <xs:restriction base="xs:date">
      <xs:minInclusive value="1980-01-01"/>
      <xs:maxInclusive value="2010-12-31"/>
    </xs:restriction>
  </xs:simpleType>

  <xs:simpleType name="sexType">
    <xs:restriction base="xs:string">
      <xs:enumeration value="M"/>
      <xs:enumeration value="F"/>
    </xs:restriction>
  </xs:simpleType>

  <!-- ATHLETE osszetett tipus -->
  <xs:complexType name="athleteType">
    <xs:sequence>
      <xs:element name="name">
        <xs:complexType>
          <xs:sequence>
            <xs:element name="first_name" type="xs:string"/>
            <xs:element name="last_name" type="xs:string"/>
          </xs:sequence>
        </xs:complexType>
      </xs:element>
      <xs:element name="birth_date" type="dateType"/>
      <xs:element name="contacts">
        <xs:complexType>
          <xs:sequence>
            <xs:element name="e_mail" type="xs:string"/>
            <xs:element name="phone_number" type="xs:string"/>
          </xs:sequence>
        </xs:complexType>
      </xs:element>
      <xs:element name="skills" type="xs:string" minOccurs="0" maxOccurs="unbounded"/>
    </xs:sequence>
    <xs:attribute name="athleteID" type="xs:integer" use="required"/>
  </xs:complexType>

  <!-- COACH osszetett tipus -->
  <xs:complexType name="coachType">
    <xs:sequence>
      <xs:element name="name">
        <xs:complexType>
          <xs:sequence>
            <xs:element name="first_name" type="xs:string"/>
            <xs:element name="last_name" type="xs:string"/>
          </xs:sequence>
        </xs:complexType>
      </xs:element>
      <xs:element name="experience_years" type="xs:integer"/>
      <xs:element name="specializations">
        <xs:complexType>
          <xs:sequence>
            <xs:element name="specialization" type="xs:string" minOccurs="1" maxOccurs="unbounded"/>
          </xs:sequence>
        </xs:complexType>
      </xs:element>
    </xs:sequence>
    <xs:attribute name="coachID" type="xs:integer" use="required"/>
  </xs:complexType>

  <!-- FACILITY osszetett tipus -->
  <xs:complexType name="facilityType">
    <xs:sequence>
      <xs:element name="name" type="xs:string"/>
      <xs:element name="address">
        <xs:complexType>
          <xs:sequence>
            <xs:element name="post_code" type="xs:string"/>
            <xs:element name="city" type="xs:string"/>
            <xs:element name="street" type="xs:string"/>
            <xs:element name="number" type="xs:string"/>
          </xs:sequence>
        </xs:complexType>
      </xs:element>
      <xs:element name="capacity" type="xs:integer"/>
    </xs:sequence>
    <xs:attribute name="facilityID" type="xs:integer" use="required"/>
  </xs:complexType>

  <!-- SPORT osszetett tipus -->
  <xs:complexType name="sportType">
    <xs:sequence>
      <xs:element name="name" type="xs:string"/>
      <xs:element name="category" type="xs:string"/>
      <xs:element name="is_indoor" type="xs:boolean"/>
    </xs:sequence>
    <xs:attribute name="sportID" type="xs:integer" use="required"/>
  </xs:complexType>

  <!-- TRAININGGROUP osszetett tipus -->
  <xs:complexType name="trainingGroupType">
    <xs:sequence>
      <xs:element name="name" type="xs:string"/>
      <xs:element name="level" type="xs:string"/>
      <xs:element name="schedule">
        <xs:complexType>
          <xs:sequence>
            <xs:element name="session" minOccurs="1" maxOccurs="unbounded">
              <xs:complexType>
                <xs:attribute name="day" type="xs:string" use="required"/>
                <xs:attribute name="from" type="xs:string" use="required"/>
                <xs:attribute name="to" type="xs:string" use="required"/>
              </xs:complexType>
            </xs:element>
          </xs:sequence>
        </xs:complexType>
      </xs:element>
    </xs:sequence>
    <xs:attribute name="groupID" type="xs:integer" use="required"/>
    <xs:attribute name="coachRef" type="xs:integer" use="required"/>
    <xs:attribute name="facilityRef" type="xs:integer" use="required"/>
    <xs:attribute name="sportRef" type="xs:integer" use="required"/>
  </xs:complexType>

  <!-- MEMBERSHIP osszetett tipus (N:M kapcsolat) -->
  <xs:complexType name="membershipType">
    <xs:sequence>
      <xs:element name="start_date" type="xs:date"/>
      <xs:element name="status" type="xs:string"/>
      <xs:element name="fee" type="xs:integer" minOccurs="0"/>
    </xs:sequence>
    <xs:attribute name="membershipID" type="xs:integer" use="required"/>
    <xs:attribute name="athleteRef" type="xs:integer" use="required"/>
    <xs:attribute name="groupRef" type="xs:integer" use="required"/>
  </xs:complexType>

  <!-- GYOKERELEM komplex tipus -->
  <xs:complexType name="clubType">
    <xs:sequence>
      <xs:element name="Athlete" type="athleteType" minOccurs="1" maxOccurs="unbounded"/>
      <xs:element name="Coach" type="coachType" minOccurs="1" maxOccurs="unbounded"/>
      <xs:element name="Facility" type="facilityType" minOccurs="1" maxOccurs="unbounded"/>
      <xs:element name="Sport" type="sportType" minOccurs="1" maxOccurs="unbounded"/>
      <xs:element name="TrainingGroup" type="trainingGroupType" minOccurs="1" maxOccurs="unbounded"/>
      <xs:element name="Membership" type="membershipType" minOccurs="0" maxOccurs="unbounded"/>
    </xs:sequence>
  </xs:complexType>

  <!-- GYOKERELEM definicioja -->
  <xs:element name="Sport_Club_VTJ4ES" type="clubType">

    <!-- KEY-k -->
    <xs:key name="AthleteKey">
      <xs:selector xpath="Athlete"/>
      <xs:field xpath="@athleteID"/>
    </xs:key>

    <xs:key name="CoachKey">
      <xs:selector xpath="Coach"/>
      <xs:field xpath="@coachID"/>
    </xs:key>

    <xs:key name="FacilityKey">
      <xs:selector xpath="Facility"/>
      <xs:field xpath="@facilityID"/>
    </xs:key>

    <xs:key name="SportKey">
      <xs:selector xpath="Sport"/>
      <xs:field xpath="@sportID"/>
    </xs:key>

    <xs:key name="TrainingGroupKey">
      <xs:selector xpath="TrainingGroup"/>
      <xs:field xpath="@groupID"/>
    </xs:key>

    <xs:key name="MembershipKey">
      <xs:selector xpath="Membership"/>
      <xs:field xpath="@membershipID"/>
    </xs:key>

    <!-- KEYREF-ek (idegen kulcsok) -->
    <xs:keyref name="TrainingGroupCoachRef" refer="CoachKey">
      <xs:selector xpath="TrainingGroup"/>
      <xs:field xpath="@coachRef"/>
    </xs:keyref>

    <xs:keyref name="TrainingGroupFacilityRef" refer="FacilityKey">
      <xs:selector xpath="TrainingGroup"/>
      <xs:field xpath="@facilityRef"/>
    </xs:keyref>

    <xs:keyref name="TrainingGroupSportRef" refer="SportKey">
      <xs:selector xpath="TrainingGroup"/>
      <xs:field xpath="@sportRef"/>
    </xs:keyref>

    <xs:keyref name="MembershipAthleteRef" refer="AthleteKey">
      <xs:selector xpath="Membership"/>
      <xs:field xpath="@athleteRef"/>
    </xs:keyref>

    <xs:keyref name="MembershipGroupRef" refer="TrainingGroupKey">
      <xs:selector xpath="Membership"/>
      <xs:field xpath="@groupRef"/>
    </xs:keyref>

  </xs:element>
</xs:schema>
\end{lstlisting}

\chapter{II. feladat - DOM}

\section{Adatolvasás}

A DOM alapú adatolvasó program Java nyelven készült, és a \texttt{Sport\_Club\_VTJ4ES.xml} állományt dolgozza fel. A DOM parser a teljes XML dokumentumot memóriába tölti, így az egyes elemekhez gyorsan és kényelmesen hozzáférhetünk.

A program lépései:

\begin{itemize}
	\item \texttt{DocumentBuilderFactory} és \texttt{DocumentBuilder} példányosítása.
	\item A \texttt{Sport\_Club\_VTJ4ES.xml} beolvasása és normalizálása.
	\item Az egyes entitáslisták beolvasása külön metódusokban: \texttt{readAthletes()}, \texttt{readCoaches()}, \texttt{readFacilities()}, \texttt{readSports()}, \texttt{readTrainingGroups()}, \texttt{readMemberships()}.
	\item A kiírás XML-szerű formátumban történik a konzolra.
\end{itemize}

\begin{lstlisting}[caption={DOMReadVTJ4ES.java} adatolvasó program, language=Java]
import javax.xml.parsers.*;
import org.xml.sax.SAXException;
import org.w3c.dom.*;
import java.io.*;

public class DOMReadVTJ4ES {

  public static void main(String[] args) {
    try {
      DocumentBuilderFactory factory = DocumentBuilderFactory.newInstance();
      DocumentBuilder builder = factory.newDocumentBuilder();
      Document document = builder.parse(
        new File("C:\\projects\\VTJ4ES_XMLGyak\\XMLTaskVTJ4ES\\Sport_Club_VTJ4ES.xml")
      );

      document.getDocumentElement().normalize();
      System.out.println("<?xml version=\"1.0\" encoding=\"UTF-8\"?>\n");
      System.out.println("<Sport_Club_VTJ4ES>\n");

      readAthletes(document);
      readCoaches(document);
      readFacilities(document);
      readSports(document);
      readTrainingGroups(document);
      readMemberships(document);

      System.out.println("\n</Sport_Club_VTJ4ES>");
    } catch (ParserConfigurationException | IOException | SAXException e) {
      e.printStackTrace();
    }
  }

  private static void readAthletes(Document document) {
    NodeList list = document.getElementsByTagName("Athlete");
    for (int i = 0; i < list.getLength(); i++) {
      Node node = list.item(i);
      if (node.getNodeType() == Node.ELEMENT_NODE) {
        Element e = (Element) node;
        String id = e.getAttribute("athleteID");

        Element nameElem = (Element) e.getElementsByTagName("name").item(0);
        String firstName = nameElem.getElementsByTagName("first_name").item(0).getTextContent();
        String lastName  = nameElem.getElementsByTagName("last_name").item(0).getTextContent();

        String birthDate = e.getElementsByTagName("birth_date").item(0).getTextContent();

        Element contacts = (Element) e.getElementsByTagName("contacts").item(0);
        String email  = contacts.getElementsByTagName("e_mail").item(0).getTextContent();
        String phone  = contacts.getElementsByTagName("phone_number").item(0).getTextContent();

        System.out.println("  <Athlete athleteID=\"" + id + "\">");
        System.out.println("    <name>");
        printElement("first_name", firstName, 6);
        printElement("last_name", lastName, 6);
        System.out.println("    </name>");
        printElement("birth_date", birthDate, 4);
        System.out.println("    <contacts>");
        printElement("e_mail", email, 6);
        printElement("phone_number", phone, 6);
        System.out.println("    </contacts>");

        NodeList skills = e.getElementsByTagName("skills");
        for (int j = 0; j < skills.getLength(); j++) {
          Node s = skills.item(j);
          if (s.getNodeType() == Node.ELEMENT_NODE) {
            printElement("skills", s.getTextContent(), 4);
          }
        }

        System.out.println("  </Athlete>");
      }
    }
  }

  private static void readCoaches(Document document) {
    NodeList list = document.getElementsByTagName("Coach");
    for (int i = 0; i < list.getLength(); i++) {
      Node node = list.item(i);
      if (node.getNodeType() == Node.ELEMENT_NODE) {
        Element e = (Element) node;
        String id = e.getAttribute("coachID");

        Element nameElem = (Element) e.getElementsByTagName("name").item(0);
        String firstName = nameElem.getElementsByTagName("first_name").item(0).getTextContent();
        String lastName  = nameElem.getElementsByTagName("last_name").item(0).getTextContent();
        String exp       = e.getElementsByTagName("experience_years").item(0).getTextContent();

        System.out.println("  <Coach coachID=\"" + id + "\">");
        System.out.println("    <name>");
        printElement("first_name", firstName, 6);
        printElement("last_name", lastName, 6);
        System.out.println("    </name>");
        printElement("experience_years", exp, 4);

        NodeList specsList = ((Element) e
            .getElementsByTagName("specializations").item(0))
            .getElementsByTagName("specialization");

        System.out.println("    <specializations>");
        for (int j = 0; j < specsList.getLength(); j++) {
          Element se = (Element) specsList.item(j);
          printElement("specialization", se.getTextContent(), 6);
        }
        System.out.println("    </specializations>");
        System.out.println("  </Coach>");
      }
    }
  }

  private static void readFacilities(Document document) {
    NodeList list = document.getElementsByTagName("Facility");
    for (int i = 0; i < list.getLength(); i++) {
      Node node = list.item(i);
      if (node.getNodeType() == Node.ELEMENT_NODE) {
        Element e = (Element) node;
        String id = e.getAttribute("facilityID");
        String name = e.getElementsByTagName("name").item(0).getTextContent();
        String capacity = e.getElementsByTagName("capacity").item(0).getTextContent();

        Element addr = (Element) e.getElementsByTagName("address").item(0);
        String post  = addr.getElementsByTagName("post_code").item(0).getTextContent();
        String city  = addr.getElementsByTagName("city").item(0).getTextContent();
        String street= addr.getElementsByTagName("street").item(0).getTextContent();
        String num   = addr.getElementsByTagName("number").item(0).getTextContent();

        System.out.println("  <Facility facilityID=\"" + id + "\">");
        printElement("name", name, 4);
        System.out.println("    <address>");
        printElement("post_code", post, 6);
        printElement("city", city, 6);
        printElement("street", street, 6);
        printElement("number", num, 6);
        System.out.println("    </address>");
        printElement("capacity", capacity, 4);
        System.out.println("  </Facility>");
      }
    }
  }

  private static void readSports(Document document) {
    NodeList list = document.getElementsByTagName("Sport");
    for (int i = 0; i < list.getLength(); i++) {
      Node node = list.item(i);
      if (node.getNodeType() == Node.ELEMENT_NODE) {
        Element e = (Element) node;
        String id    = e.getAttribute("sportID");
        String name  = e.getElementsByTagName("name").item(0).getTextContent();
        String cat   = e.getElementsByTagName("category").item(0).getTextContent();
        String indoor= e.getElementsByTagName("is_indoor").item(0).getTextContent();

        System.out.println("  <Sport sportID=\"" + id + "\">");
        printElement("name", name, 4);
        printElement("category", cat, 4);
        printElement("is_indoor", indoor, 4);
        System.out.println("  </Sport>");
      }
    }
  }

  private static void readTrainingGroups(Document document) {
    NodeList list = document.getElementsByTagName("TrainingGroup");
    for (int i = 0; i < list.getLength(); i++) {
      Node node = list.item(i);
      if (node.getNodeType() == Node.ELEMENT_NODE) {
        Element e = (Element) node;
        String id        = e.getAttribute("groupID");
        String coachRef  = e.getAttribute("coachRef");
        String facRef    = e.getAttribute("facilityRef");
        String sportRef  = e.getAttribute("sportRef");
        String name      = e.getElementsByTagName("name").item(0).getTextContent();
        String level     = e.getElementsByTagName("level").item(0).getTextContent();

        System.out.println("  <TrainingGroup groupID=\"" + id
                           + "\" coachRef=\"" + coachRef
                           + "\" facilityRef=\"" + facRef
                           + "\" sportRef=\"" + sportRef + "\">");
        printElement("name", name, 4);
        printElement("level", level, 4);

        Element sched = (Element) e.getElementsByTagName("schedule").item(0);
        NodeList sessions = sched.getElementsByTagName("session");
        System.out.println("    <schedule>");
        for (int j = 0; j < sessions.getLength(); j++) {
          Element s = (Element) sessions.item(j);
          String day  = s.getAttribute("day");
          String from = s.getAttribute("from");
          String to   = s.getAttribute("to");
          System.out.println("      <session day=\"" + day
                             + "\" from=\"" + from
                             + "\" to=\"" + to + "\"/>");
        }
        System.out.println("    </schedule>");
        System.out.println("  </TrainingGroup>");
      }
    }
  }

  private static void readMemberships(Document document) {
    NodeList list = document.getElementsByTagName("Membership");
    for (int i = 0; i < list.getLength(); i++) {
      Node node = list.item(i);
      if (node.getNodeType() == Node.ELEMENT_NODE) {
        Element e = (Element) node;
        String id     = e.getAttribute("membershipID");
        String athRef = e.getAttribute("athleteRef");
        String grpRef = e.getAttribute("groupRef");
        String start  = e.getElementsByTagName("start_date").item(0).getTextContent();
        String status = e.getElementsByTagName("status").item(0).getTextContent();
        String fee    = "";
        if (e.getElementsByTagName("fee").getLength() > 0) {
          fee = e.getElementsByTagName("fee").item(0).getTextContent();
        }

        System.out.println("  <Membership membershipID=\"" + id
                           + "\" athleteRef=\"" + athRef
                           + "\" groupRef=\"" + grpRef + "\">");
        printElement("start_date", start, 4);
        printElement("status", status, 4);
        if (!fee.isEmpty()) {
          printElement("fee", fee, 4);
        }
        System.out.println("  </Membership>");
      }
    }
  }

  private static void printElement(String name, String content, int indent) {
    StringBuilder sb = new StringBuilder();
    for (int i = 0; i < indent; i++) sb.append(" ");
    sb.append("<").append(name).append(">")
      .append(content)
      .append("</").append(name).append(">");
    System.out.println(sb.toString());
  }
}
\end{lstlisting}

\section{Adatmódosítás}

Az adatmódosító program (DOMModifyVTJ4ES.java) szintén DOM API-t használ. Például a következő módosításokat hajtja végre:

\begin{itemize}
	\item Minden \texttt{Athlete} \texttt{athleteID} attribútuma elé \texttt{"ATH\_"} előtag kerül.
	\item Minden \texttt{TrainingGroup} esetén növeli a logikai befogadóképességet egy \texttt{max\_athletes} attribútummal, fix \texttt{30} értékre.
	\item A \texttt{Membership} elemek közül az aktív tagságnál (status = \texttt{active}) a \texttt{fee} elemet 10\%-kal megemeli.
\end{itemize}

(A forráskód szerkezete megegyező a mintában látott \texttt{DOMModifyKLNSPG} felépítésével, csak a sport club elemekre és attribútumokra van átírva.)

\section{Adatlekérdezés}

Az adatlekérdező program (DOMQueryVTJ4ES.java) különböző szempontok szerint gyűjti ki az XML-ből az adatokat, például:

\begin{itemize}
	\item Férfi sportolók listázása (ha a sémában szerepel nem szerinti jelölés).
	\item A legnagyobb kapacitású \texttt{Facility} kiválasztása.
	\item 2000 után született \texttt{Athlete}-ek lekérdezése.
	\item Egy kiválasztott \texttt{TrainingGroup} összes tagságának (\texttt{Membership}) listázása.
\end{itemize}

A kód felépítése hasonló a kapott DOMQuery példához: \texttt{DocumentBuilder}-rel beolvassa a dokumentumot, majd \texttt{getElementsByTagName} és attribútumolvasás segítségével készíti el az XML-szerű kimenetet.

\section{Adatírás}

Az adatíró program (DOMWriteVTJ4ES.java) a memóriában lévő DOM fát írja ki egy új XML fájlba (pl. \texttt{Sport\_Club\_VTJ4ES\_output.xml}). A \texttt{Transformer} osztály segítségével gondoskodik az olvasható, behúzásokkal formázott kimenetről, hasonlóan a mintában bemutatott \texttt{DOMWriteKLNSPG} programhoz.

\end{document}
